\chapter{Creating a Successful Group}
While Experts in Teamwork frequently focuses on dealing with the challenges of working in cross-disciplinary groups, we also looked at some prerequisites for creating successful groups in the first place. Some of the measures we took internally to ensure a well-functioning group are described below. 
\section{Group Goal}
According to Johnson et al.~\cite{2013johnson} effective groups state clear goals and strategies to achieve those goals. At the start of the project, our team agreed on a common goal - achieving the best possible grade. We also aimed to create a substantial prototype which could potentially be added to our own portfolios. During first meetings we spent a lot of time trying to reach a common understanding of the game we were developing and how we could achieve our goals. We divided tasks among team members close to their interest which let us be committed to the group goal. 
\section{Group Rules}
Another characteristics of effective groups which is mentioned by Johnson et al.~\cite{2013johnson} is effective communication to reduce misunderstandings. 
Our group contract dictated that we should use Discord\footnote{\url{https://discordapp.com/}} for communication, which we were required to check daily. We also defined teamwork climate rules, like how criticism should always be constructive. 
We also stated that if someone had a problem regarding our project this person should inform members as soon as possible and as that we should try to help as a group. 

Equal participation, shared leadership and members’ power based on expertise, ability and access to information is the next characteristic mentioned by Johnson et al.~\cite{2013johnson}.
In our group contract we agreed on the decision process and group members roles.
We decided that large, impactful decisions should be made by all group members, while smaller decisions would be made by the technical leader and art director.
