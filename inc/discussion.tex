\chapter{Discussion} \label{chap:discussion}
\section{Reasons for group success}
Although our group was not entirely free from problems, the project as a whole can be characterized by good cooperation and employment of good practices, resulting in what we would describe as a successful project. This section discusses what we believe to be the reasons behind our success.
\subsection{Diversity}
Diversity can be beneficial to the group performance. As mentioned by Johnson and Johnson~\cite[p. 445]{2013johnson}:
\begin{quote}
Diverse groups tend to be more creative in their problem solving than are homogenous groups. The conflicts and disagreements that arise from the different perspectives and conclusions generate more creativity than is available in homogenous groups.
\end{quote}

As demonstrated in the “triangle exercise” (see section ~\ref{sec:assigning_roles}) our group possess diverse abilities and skills. Teamwork, in most cases, require a multi-disciplinary approach, and its safe to say that if we did no have a diverse team we would not have been nearly as successful as we were. Furthermore, our knowledge and skills from different fields this made it easier to divide tasks between members, reducing potential conflicts. We trusted each others expertise and abilities when working with the game. For example, when dealing with the musical part of our game team members without a musical background, Rikhart and Yijie, trusted the abilities of the rest of the team members. Similarly, group members with design backgrounds were trusted to make the correct decisions on design related problems. We believe that trusting each-others abilities decreased conflict.  

Johnson and Johnson also mentioned~\cite[p. 446]{2013johnson}:
\begin{quote}
\quote{Diversity creates difficulties in communication, coordination, and decision making. These difficulties result in spending more time trying to communicate and less time completing the task.}
\end{quote} 
This, combined with their backgrounds in programming and design, is why both Sabina and Rikhart were given role of cross-discipline communicators. They were both also flexible, allowing them to contribute both to design and to programming.

Our group consisted of 4 males and 2 females, from three different countries. This required us to pay greater attention to how we used language to formulate our ideas and opinions, in an understandable fashion. An interesting observation was made by Yijie regarding how in his culture the feelings of group members is less important than the work being done. This is quite different to the Norwegian model, where individuals’ personal feelings and contributions are considered more important in group collaboration. 

Our group had four Norwegian members, therefore, Sabina and Yijie could observe the Norwegian style of working. One example was the strong emphasis on working together at the same time with everyone engaged in the same task. At the first glance this might look ineffective, while in reality it helped keeping everyone involved and created a common vision. 

\subsection{Need for Affiliation}
Chun and Choi~\cite{2014chin_and_choi} speaks of the need for affiliation (a need to belong) as one of three needs in their study. The others being control and achievement. We identified that this need was an attribute all of the members possessed and had fairly equal amounts of. As Chun and Choi~\cite{2014chin_and_choi} points out, the presence of this need plays a role in the groups performance, but so does the strength of this need and how it is distributed amongst the members as well. They further explain that if one member has a strong need for affiliation this might cause conflict, while if the distribution is equal, or close to equal, it does not cause problems. This is because there is a higher understanding amongst the groups members that everyone has the same goal in the group - to belong. 
We believe that our group had an equal distribution of the affiliation need and that this contributed to our productivity and lack of conflict.

\subsection{Communication}
Another self-identified key for success was good intragroup communication. The group decided to make communication a priority, as several of the group members had negative previous experiences with poor communication in group work. 

Johnson and Johnson~\cite[p. 44]{2013johnson} lists \begin{quote} ...accurate and complete communication among members \end{quote} as one of the requirements for an effective group. In the aforementioned previous groups there were an imbalance amongst the different members need for information and the comfort of knowing the current progress. This resulted in some members not communicating at all. A whole range of issues ensued from this one problem, causing these groups to be very ineffective.  

We therefore made it a point, even as early as when forming the group contract, to include certain stipulations directly related to group communication. These group rules included demands to check Discord\footnote{\url{https://discordapp.com/}} every day and also to notify about changes in plans as soon as possible. This meant that if someone couldn’t meet, adjustments could be made in time. The use of the Discord platform allowed for instant communication if needed. Since the contract stated that each member had to check this platform at least once a day, when someone wanted opinions or help, they could ask over discord and receive fast answers.
For example, this happened when Ellinor wanted opinions on her graphic design for the user interface and wanted to know if she could move forward with it, she asked via Discord and received fast answers. This eliminated a lot of waiting, allowing Ellinor to proceed quickly to the next step.

Another interesting aspect to our communication was that while it was highly task related, it was not exclusively task related. The importance of topical balance has been covered by Wheelan~\cite{wheelan} who points out that in effective teams the ratio of task and goal related communication is usually around 70-80\%. This ratio seems reasonably close to our communication pattern. While we were heavily task focused, we would often change to a non work related topic after longer intense periods of task oriented discussions.

\subsection{High mutual dependency and shared responsibility}
In his book “Team”, Kjell B. Hjertø defines a team as a work group of at least 3 members that have a high degree of mutual dependency and a large amount of shared responsibility in order to satisfy the group’s goals~\cite{team_book}. The balance between mutual dependency and shared responsibility is necessary for a team to function optimally.

Related to our group, we think that there was indeed high mutual dependency and high shared responsibility. Our mutual dependency came from the fact that we each had different specialties and each of us worked on different important components of the game that depended on each other. As a result, we also had high shared responsibility, as everyone is working on different and essential components of the project. One of the ways we can see our shared responsibility is in how we don’t assign blame, which is something that often happen in ineffective teams~\cite{wheelan}. If something doesn’t work it is the whole groups problem, not just a single member’s.

\subsection{High Task Focus}
Throughout the project we have had a large focus on team productivity, efficiency and the task at hand. As mentioned in the project report we chose methodologies and workflows that removed what we perceived as unnecessary overhead or distractions. We see this focus as a contributor to our groups success, which also aligns with Wheelans~\cite{wheelan} guideline of encouraging and supporting norms that promote effectiveness and productivity. 

\section{Situations}
In context of the EiT course, the term “situation” can be understood as a series of circumstances which lead to misunderstandings and disagreements. The “situation” creates high tension between group members, potentially leading to an argument or even a conflict. The “situations” can originate from differing opinions, non-conformity, lack of common understanding of domain or goal, lack of dialog, different members orientations (task-, relationship -, individually oriented people), or low level of transparency~\cite{ntnu_eit_book}.
\subsection{Audio recording}
The group had a bit of a split when we discussed how should we record the group meetings on village day 3. Per-Morten suggested doing audio recordings of group meetings because it would be easy to look back at specific times and clearly hear what people mentioned or thought about the discussion. Sabina mentioned that it would be uncomfortable to record everything and Rikhart agreed with this as it would be better to focus on the reflections at the moment and not necessarily too much in retrospect.

This situation occurred due to the diversity of our personalities and perspectives. Per-Morten tended to approach the situation in a logical way, he focused on the facts about the decisions. Meanwhile, Sabina saw the situation from an emotional perspective. 
Additionally this situation was partly due to misunderstandings of whether or not the recording was a course requirement. Yijie thought we were required to record the process using videos or audios and preferred to do audio recording if this was a prerequisite.  

Diversity can lead to either positive or negative outcomes in group, which highly depends on group members' willingness to understand and appreciate the existing diversity in the group~\cite{2013johnson}. Since this situation happened on village day 3, there was still an unfamiliarity between the members within the group. To handle this situation, we had an open discussion about whether or not we should do the audio recording. Ellinor made sure that everyone was able to express their opinions.
The discussion helped us get a better understanding of each other. We realized there was diversity in the group and we tried to respect and value everyone’s view. This allowed us to view the situation in a holistic way, and knowing each other’s opinions further helped us discover future potential misunderstandings.

After the discussion, the situation was resolved and we reached an agreement to use the daily group reflections to document the process. This way we ensured a precise recording of the information without any of the group members feeling uncomfortable. 

\subsection{Cross-disciplinary communication}
On day 3, the group experienced a second situation. After making an initial time schedule for the project, we decided that we needed to refine the initial project idea, before we could begin actual work on either design or code. While Per-Morten and Andreas were keen on initially defining base functionalities, Sabina and Ellinor wanted to discuss overarching themes and user experiences. Both groups defined this as describing the “basic concept”, although they were really thinking about quite different aspects of the project. 

Roger Schwarz~\cite{scharz} describes a set of “ground rules”; a set of behaviours that can be used to identify effective groups, or improve group process. The situation described above can be directly linked to his third ground rule; \textit{Use specific examples and agree on what important words mean}. Agreeing on what important words means within a group can help avoid misunderstandings that otherwise are bound to happen. Using specific examples can be helpful to describe the meaning of words, as well as what they do not mean if necessary. While the discussion went on for a while during the meeting, it was first when we started giving specific examples of what we meant by saying “basic concept” that the problem was identified. 

Several times during the project we would encounter similar situations. However, the group’s first discussion on “basic concept” had helped us recognize what was happening when the situations occurred. By taking a step back and explaining our understanding of the words that we used, we managed to resolve the situations quickly. The original situation seems to have helped us become more aware of communication challenges across disciplines, and better prepared to recognize and handle similar situations.

\subsection{Imbalance among talkers}
Early on in the process it became apparent that there was an imbalance of who was talking and leading the discussions. This is an undesirable feature, as open communication with inputs from all members is a factor that often identifies effective teams~\cite{wheelan}. In our group Per-Morten and Ellinor were dominating the discussions. The facilitator presented the group with a sociogram confirming this.
This issue stayed with us for the duration of the project, but we managed to negate it through a few deliberate actions.

In the beginning, this imbalance was likely partly due to the fact that we were cautious with each other, since we did not know each other very well yet. Sabina admitted that she was careful when it came to speaking in groups she did not feel comfortable in or knew well. 

To ensure that Ellinor and Per-Morten would not dominate all the conversations, and to involve the others and get their thoughts and input, Ellinor and Per-Morten tried to halt and ask for the others opinion when they felt they were talking too much. We saw this as a necessary solution, since part of the problem was that the rest of the members did not interject to point this out of their own accord. We also decided to have a 3 minutes stand-up meeting in the morning were everyone could speak. By forcing it to be maximum 3 minutes for each person, we made sure the two main speakers could not speak for longer than anyone else. This created a better balance amongst us.      

Chun and Choi~\cite{2014chin_and_choi} showed that the need of each group member has an impact on the groups performance. They studied three needs that group members might have within a group. One of these is the need for power. Although it seemed at first that Ellinor and Per-Morten might have this need; which would explain why they would always dominate and steer the direction of the conversations, other properties made us deem this as unlikely. Most of all because neither seemed to have any problems yielding when they needed to, and also because both agreed that they were speaking too much compared to the rest. Rather than something as strong as a \textit{need}, our theory is that their personality naturally lent itself to verbalizing and sharing their ideas and thoughts a lot. While the personality of the other members naturally let this go unobstructed, the other members were just happy to let someone else speak.

\subsection{Allocating time for project work}
The fourth situation happened when we concentrated on the project, the flow of work was often interrupted by the course exercises. For example, we expected to work on the project on day 6, but as a part of the course requirement, we had to review the group contract in the middle of the day. Per-Morten felt that this frequent switching of context was disruptive and made it hard for him to focus on the project and be productive.

This situation happened partly due to the nature of the course, which required us to both work on the project and have training in teamwork at the same time. As we could not change the course components, we had to adjust our schedule to improve this situation. Normally the plan of the coming village day was published one day in advance. We asked the supervisor McCallum, the lecturer of the course, to publish the plan earlier so we could have enough time to schedule work sessions in advance. We allocated a block of time to focus on the project each day, and during that period, we tried to avoid being interrupted by long winded discussions or course exercises.
 
Another reason for this situation was that all the group members felt strongly motivated to work on the project itself, rather than doing the course exercises. This may have been because we all have a high need for achievement~\cite{2014chin_and_choi}. A group with members who demonstrate a high level of need for achievement on average may be strongly task oriented and desire for superior performance~\cite{2014chin_and_choi}. When we encountered distractions, we lost our attention on the current task, which led to an outcome of lower quality. We therefore felt more sensitive and annoyed when the work disturbed by the course exercises. 

Allocating time properly helped increase motivation in the group, both for the project and for the course exercises. During the allocated time period for group work, we felt more concentrated and efficient.
